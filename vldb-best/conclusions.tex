\section{Conclusions}\label{sec:conclusions}%\label{sec:ddlog}

\subsection{Adoption}

Traditional databases could in principle be retrofitted to use the
algorithms in this paper, but the existing query engines are not built
around structures that can represent negative changes (like \zrs), so
this effort will require a significant redesign.

Moreover, we argue that databases should not only compute views
incrementally, but should use ``changes'' as the fundamental data
structure to communicate with their environment: a database service
should offer the following API: users register to receive
notifications for changes in one or more views.  Then, for any
transaction committed, each user receives a notification containing
the list of changes for the all the views they registered.  Databases
today do not have convenient mechanism for reporting changes to the
outside world.  In fact, entire industries have sprung up around the
concept of Change Data Capture~\cite{cdc}, which is building ad-hoc
solutions for extracting changes from databases, usually by inspecting
the write-ahead transaction log.

\subsection{Summary}

We have introduced \dbsp, a model of computation based on infinite
streams over commutative groups.  In this model streams are used for 3
different purposes: (1) to model consecutive snapshots of a database,
(2) to model consecutive changes (deltas, or transactions) applied to
a database and changes of a maintained view, (3) to model consecutive
values of loop-carried variables in recursive computations.

We have defined an abstract notion of incremental computation over
streams, and defined the incrementalization operator $\inc{\cdot}$,
which transforms an \emph{arbitrary} stream computation $Q$ into its
incremental version $\inc{Q}$.  The incrementalization operator has
some very nice algebraic properties, which gave us a general algorithm
for incrementalizing many classes of complex queries, including
arbitrary recursive queries.

We believe that \dbsp can form a solid foundation for a theory and
practice of streaming incremental computation.  As a proof, we have
built a SQL compiler that can essentially incrementalize arbitrary
queries.
