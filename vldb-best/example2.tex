\subsection{Example recursive query}\label{sec:recursive-example}

Let us apply the algorithm~\ref{algorithm-rec}, that converts a
recursive query into a DBSP circuit, to a concrete Datalog program,
which computes the transitive closure of a directed graph:

\begin{lstlisting}[language=ddlog,basicstyle=\small]
// Edge relation with head and tail
input relation E(h: Node, t: Node)
// Reach relation with source s and sink t
output relation R(s: Node, t: Node)
R(x, y) :- E(x, y).
R(x, y) :- E(x, z), R(z, y).
\end{lstlisting}

Step 1: we ignore the fact that R is both an input and an output and we implement
the \dbsp circuit corresponding to the body of the query.  This query could be expressed
in SQL as:

\begin{lstlisting}[language=SQL,basicstyle=\small]
( SELECT * FROM E)
UNION
( SELECT E.h , R.t
  FROM E JOIN R
  ON E.t = R.s )
\end{lstlisting}

This query generates a \dbsp circuit with inputs \code{E} and \code{R}:

\begin{tikzpicture}[>=latex, node distance=1.2cm]
  \node[] (E) {\code{E}};
  \node[above of=E, node distance=.6cm] (R1) {\code{R}};
  \node[block, right of=R1] (j) {$\bowtie_{t=s}$};
  \node[block, right of=j] (pj) {$\pi_{h, t}$};
  \node[block, circle, below of=pj, inner sep=0cm, node distance=.6cm] (plus) {$+$};
  \node[block, right of=plus] (d) {$\distinct$};
  \node[right of=d] (R) {\code{R}};
  \draw[->] (R1) -- (j);
  \draw[->] (E) -- (j);
  \draw[->] (j) -- (pj);
  \draw[->] (E) -- (plus);
  \draw[->] (pj) -- (plus);
  \draw[->] (plus) -- (d);
  \draw[->] (d) -- (R);
\end{tikzpicture}

Step 2: Lift the circuit by lifting each operator pointwise:

\begin{tikzpicture}[>=latex, node distance=1.4cm]
  \node[] (E) {\code{E}};
  \node[above of=E, node distance=.6cm] (R1) {\code{R}};
  \node[block, right of=R1] (j) {$\lift{\bowtie_{t=s}}$};
  \node[block, right of=j] (pj) {$\lift{\pi_{h, t}}$};
  \node[block, circle, below of=pj, node distance=1cm, inner sep=0cm] (plus) {$+$};
  \node[block, right of=plus] (d) {$\lift{\distinct}$};
  \node[right of=d] (R) {\code{R}};
  \draw[->>] (R1) -- (j);
  \draw[->>] (E) -- (j);
  \draw[->>] (j) -- (pj);
  \draw[->>] (E) -- (plus);
  \draw[->>] (pj) -- (plus);
  \draw[->>] (plus) -- (d);
  \draw[->>] (d) -- (R);
\end{tikzpicture}

Step 3: Connect the feedback loop implied by relation \code{R}:

\begin{tikzpicture}[>=latex]
  \node[] (E) {\code{E}};
  \node[right of=E] (empty) {};
  \node[block, above of=empty, node distance=.6cm] (j) {$\lift{\bowtie_{t=s}}$};
  \node[block, right of=j, node distance=1.4cm] (pj) {$\lift{\pi_{h, t}}$};
  \node[block, circle, below of=pj, node distance=.8cm, inner sep=0cm] (plus) {$+$};
  \node[block, right of=plus, node distance=1.4cm] (d) {$\lift{\distinct}$};
  \draw[->>] (E) -- (j);
  \draw[->>] (j) -- (pj);
  \draw[->>] (E) -- (plus);
  \draw[->>] (pj) -- (plus);
  \draw[->>] (plus) -- (d);

  % generic part
  \node[right of=d, node distance=1.3cm] (output)  {\code{R}};
  \draw[->>] (d) -- node (o) {} (output);
  \node[block, above of=j, node distance=.8cm] (z) {$\zm$};
  \draw[->>] (o.center) |- (z);
  \draw[->>] (z) -- (j);
\end{tikzpicture}

Step 4: ``bracket'' the circuit, first with $\I$-$\D$, then with $\delta$-$\int$:

\noindent
\begin{tikzpicture}[>=latex, node distance=1.3cm]
  \node[] (Einput) {\code{E}};
  % generic part
  \node[block, right of=Einput, node distance=.8cm] (ID) {$\delta$};
  \node[block, right of=ID, node distance=.8cm] (E) {$\I$};

  % relational query
  \node[block, above of=E, node distance=1cm] (j) {$\lift{\bowtie_{t=s}}$};
  \node[block, right of=j, node distance=1.4cm] (pj) {$\lift{\pi_{h, t}}$};
  \node[block, circle, below of=pj, node distance=1cm, inner sep=0cm] (plus) {$+$};
  \node[block, right of=plus] (d) {$\lift{\distinct}$};
  \draw[->>] (E) -- (j);
  \draw[->>] (j) -- (pj);
  \draw[->>] (E) -- (plus);
  \draw[->>] (pj) -- (plus);
  \draw[->>] (plus) -- (d);

  % generic part
  \node[block, right of=d, node distance=1.3cm] (D) {$\D$};
  \node[block, right of=D, node distance=1cm] (S) {$\int$};
  \node[right of=S, node distance=.8cm] (output)  {\code{R}};
  \draw[->] (Einput) -- (ID);
  \draw[->>] (ID) -- (E);
  \draw[->>] (d) -- node (o) {} (D);
  \draw[->>] (D) -- (S);
  \draw[->] (S) -- (output);
  \node[block, above of=j, node distance=.8cm] (z) {$\zm$};
  \draw[->>] (o.center) |- (z);
  \draw[->>] (z) -- (j);
\end{tikzpicture}

The above circuit is a complete implementation of the non-streaming
recursive query; given an input relation \code{E} it will produce
its transitive closure \code{R} at the output.

Now we use semina\"ive evaluation to rewrite the circuit (to save
space in the figures we will omit the indices from $\pi$ and $\sigma$
in the subsequent figures):

\noindent
\begin{tikzpicture}[>=latex, node distance=1.4cm]
  \node[] (Einput) {\code{E}};
  % generic part
  \node[block, right of=Einput, node distance=.8cm] (E) {$\delta$};

  % relational query
  \node[block, above of=E, node distance=1cm] (j) {$\inc{(\lift{\bowtie})}$};
  \node[block, right of=j, node distance=1.5cm] (pj) {$\inc{(\lift{\pi})}$};
  \node[block, circle, below of=pj, node distance=1cm, inner sep=0cm] (plus) {$+$};
  \node[block, right of=plus] (d) {$\inc{(\lift{\distinct})}$};
  \draw[->>] (E) -- (j);
  \draw[->>] (j) -- (pj);
  \draw[->>] (E) -- (plus);
  \draw[->>] (pj) -- (plus);
  \draw[->>] (plus) -- (d);

  % generic part
  \node[block, right of=d, node distance=1.5cm] (S) {$\int$};
  \node[right of=S, node distance=.8cm] (output)  {\code{R}};
  \draw[->] (Einput) -- (E);
  \draw[->>] (d) -- node (o) {} (S);
  \draw[->] (S) -- (output);
  \node[block, above of=j, node distance=1cm] (z) {$\zm$};
  \draw[->>] (o.center) |- (z);
  \draw[->>] (z) -- (j);
\end{tikzpicture}

Using the linearity of $\lift\pi$, this can be rewritten as:

\noindent
\begin{tikzpicture}[>=latex, node distance=1.5cm]
  \node[] (Einput) {\code{E}};
  % generic part
  \node[block, right of=Einput, node distance=.8cm] (E) {$\delta$};

  % relational query
  \node[block, above of=E, node distance=1cm] (j) {$\inc{(\lift{\bowtie})}$};
  \node[block, right of=j] (pj) {$\lift{\pi}$};
  \node[block, circle, below of=pj, node distance=1cm, inner sep=0cm] (plus) {$+$};
  \node[block, right of=plus] (d) {$\inc{(\lift{\distinct})}$};
  \draw[->>] (E) -- (j);
  \draw[->>] (j) -- (pj);
  \draw[->>] (E) -- (plus);
  \draw[->>] (pj) -- (plus);
  \draw[->>] (plus) -- (d);

  % generic part
  \node[block, right of=d] (S) {$\int$};
  \node[right of=S, node distance=.8cm] (output)  {\code{R}};
  \draw[->] (Einput) -- (E);
  \draw[->>] (d) -- node (o) {} (S);
  \draw[->] (S) -- (output);
  \node[block, above of=j, node distance=1cm] (z) {$\zm$};
  \draw[->>] (o.center) |- (z);
  \draw[->>] (z) -- (j);
\end{tikzpicture}

