\section{Experimental Evaluation}\label{sec:experiments}

\newcommand{\query}[1]{\textsf{#1}}

In this section we quantify some aspects of our implementation.
Looking at Figure~\ref{fig:adapters}, it is clear that data crosses
many layers.  In this section we evaluate only the performance of
the central block, the \dbsp query engine.  However, in most real-life
application performance will be limited by the adapters and network.

\subsection{Cost of incremental updates}

Here we validate our claim that updates in \dbsp have a cost
proportional to the size of the change.

We ran TPC-H on a desktop computer, with a scale factor 100, meaning
that the input data is about 100 GB, consisting of about 1.6 billion
records.  We divided the largest tables, \emph{orders} and
\emph{lineitem}, which in total contain about half the input records,
into 10 equal-sized batches, each comprising about 75,000,000 records.
Because \emph{orders} records refer to \emph{lineitem} records, we
ensured that if an \emph{order} record was in a batch, so were its
\emph{lineitem} records.  Then, for each batch in turn, we inserted
its records.

We ran this computation using two database systems: (1) using a
popular open-source database\footnote{Because it's hard to make a fair
comparison, we won’t specify the database.}, by rerunning the query
after each insertion, and (2) using our incremental implementation,
where we declare the query as a view.

We show results for TPC-H query 5, which is a 6-way join that we
selected as representative of the set of TPC-H queries.  We did not
use indexes with the database, because we found that they made the
overall computation slower.

Figure~\ref{fig:tpch} shows the time taken to process each batch.  As
expected, ignoring the absolute performance, and looking just at
trends, the data shows that re-executing queries on an increasing
dataset slows down as the total data size increases, whereas
incremental computation performance remains steady.

\begin{figure}[t]
  \begin{center}
  \includegraphics[scale=.4]{graph/tpch}
  \caption{Incremental versus batch performance on TPC-H
    Q5\label{fig:tpch}.}
  \end{center}
\end{figure}

\vspace{-3ex}
\subsection{Latency, throughput, and input change size}

Latency is the time between submitting a change and obtaining a
result.  Observed latency is a function of both query complexity and
the size of the internal state, so latency will change as system
state grows.

For relatively simple queries, as described in
Section~\refsec{sec:macrobenchmarks}, while running in steady state,
the latency of a transaction changing a single input row is in the
order of tens to hundreds of microseconds, proving that our engine can
be used for very low latency applications.

Throughput is the number of records that can be processed in a time
unit.  \dbsp is synchronous and blocking: for every input
change, the pipeline does not accept any other inputs until it has
produced the output for all views.  This suggests that latency is the
inverse of throughput.

There is an additional degree of freedom: the size of an input
transaction.  In several scenarios there is a choice: (1) when a
pipeline is started and is ingesting the initial state of a large
database (\emph{backfilling}), or (2) when processing data from
streaming sources, without clear transaction boundaries.

As Nikolic has observed before~\cite{nikolic-sigmod16}, there is a
relatively tight relationship between the latency of updating a view,
the throughput, and the size of the input changes.  Nikolic finds that
in DBToaster the optimal value is somewhere between 1K and 10K tuples.
Our experiments confirm this.  The exact optimum value depends on the
query and data distribution.  Figure~\ref{fig:batchsize} shows some
typical measurements for Nexmark query \query{q5}.  Latency grows
monotonically with input batch size, but the optimum throughput is
obtained for batches of 2K-20K records.

\begin{figure}[h]
  \includegraphics[width=.90\columnwidth]{graph/batchsize}
  \caption{Latency and throughput as a function of the input batch
    size.  Notice the logarithmic X axis.\label{fig:batchsize}}
\end{figure}

\begin{comment}
Increasing the batch size further beyond 100K records causes the
system to enter an unstable state (the crossover point depends on the
available memory size), in which throughput oscillates, as shown in
Figure~\ref{fig:oscillation}.  One reason this happens is that some
activities, such as merging batches, and garbage collection of useless
records, only happen between circuit steps.  The right solution to
smooth these oscillations is probably a dynamic controller for input
sizes, but also for partitioning resources like memory between \dbsp
operators and their shards.

\begin{figure}[h]
  \begin{center}
  \includegraphics[scale=.5]{graph/oscillation}
  \caption{Unstable throughput for very large input batch
    sizes\label{fig:oscillation}.}
  \end{center}
\end{figure}
\end{comment}

\vspace{-3ex}
\subsection{Macrobenchmarks}\label{sec:macrobenchmarks}

\begin{figure*}
  \includegraphics[width=.95\textwidth]{graph/throughput} \hspace{-3ex}
  (a) \\
  \includegraphics[width=.95\textwidth]{graph/memory} \hspace{-3ex}
  (b) \\
  \caption{(a) Average normalized throughput with respect to Flink
    (higher is better), and (b) peak memory consumption (lower is
    better), in GiB (\(2^{30}\) bytes).  The workload comprises the
    Nexmark queries that \dbsp and Flink support in common, over
    100,000,000 events.  There are no measurements for q11 for
    \dbsp.\label{fig:macrobenchmark}}
\end{figure*}

There are no standard benchmark suites for IVM.  In this section we
use an atypical benchmark suite, Nexmark~\cite{tucker2008nexmark},
which was designed for benchmarking streaming systems.  The benchmark
is driven by a synthetic data generator, modeling an online
auction site along with a suite of queries against the streams.
Nexmark is already implemented for other streaming database systems,
notably for Flink~\cite{carbone-ieee15,nexmark-flink}, a widely used
stream processing system.  Nexmark is an unusual benchmark, since the
data is append-only, and thus grows unbounded.  The required internal
state would also grow unbounded if the input is assumed to be
arbitrary.  However, given some weak assumptions about the ordering of
the inputs, these queries \emph{can} be implemented using finite state
using a garbage-collection mechanism that deletes internal state which
cannot influence any future outputs.  We leave this subject for a
future paper.  This kind of benchmarks can only be implemented using a
\dbsp-like model of computation, where the output is a stream of
changes (and not the full views, which would also grow unbounded).

We compare \dbsp against Flink on the Nexmark benchmark, which
consists of 23 queries.  We show results for the queries that Flink
and \dbsp both support.  We omitted \query{q6} because there was no
Flink implementation, and \query{q10} because we could not make
Flink's implementation for it work.  \dbsp does not support
\query{q11} because \dbsp does not yet support session windows.

We ran both the Flink and \dbsp implementations on the same machine,
which has a 64-core, 128-thread Threadripper~3990X CPU and 256~GB RAM,
with Fedora Core~40 as the operating system. We present results for
100~million Nexmark events (input records), which is a moderate
number.

\dbsp runs as a single process with 16~worker threads, and otherwise
with default settings, matching the number of workers used for Flink.
We ran \dbsp both with storage disabled, where \dbsp keeps all state
in RAM, and with storage enabled, where \dbsp flushes large batches to
secondary storage (see~\ref{sec:state-management}).  Enabling storage
allows \dbsp to work with more state with less memory use, at some
cost in throughput.

We configured the Flink implementation of Nexmark with the settings
recommended by the upstream project, running 8~Flink task manager
containers, each allocated 2~cores, and one Flink job manager
container.  We tried adjusting Flink and Nexmark settings, but none of
these changes improved Flink performance in a significant and
reproducible way.

\dbsp and Flink support reading input from multiple kinds of data
sources.  For these measurements, we configured both of them to use
their own integrated Nexmark event generators, rather than pulling
them from Kafka or HTTP or another source.  This eliminated network
service performance and configuration as a possible source of
variability.

Figure~\ref{fig:macrobenchmark} reports our measurements.  We show 3
bars for \dbsp: the first (blue) is using hand-written Rust code based
on the \dbsp runtime, running in memory; the second (orange) bar is
written in SQL and runs in memory, while the third (red) bar shows the
SQL program using secondary storage.

\paragraph{Throughput.}

\newcommand{\x}{\(\times\)}

Figure~\ref{fig:macrobenchmark}(a) shows the normalized throughput of
\dbsp versus Flink (which is always 1).  With storage disabled, \dbsp
is up to 17\x{} faster than Flink, with a geometric mean of 2.2\x{}
faster on average.  As queries get more complicated, \dbsp's advantage
over Flink grows by a much larger factor.  We also notice that there
is still a roughly $2\times{}$ gap to cover between the performance of
hand-written \dbsp programs and the code generated by the SQL
compiler.

\query{q13} is an outlier that performs slightly slower in our system
than Flink; with storage enabled, it is slower than Flink.  We are
investigating this behavior.  With \query{q13} excluded, every
remaining query runs at least 1.4\x{} faster in \dbsp (with or without
storage).

Storage generally has a small impact on our throughput, except for
\query{q13}, where it has about a 4\x{} penalty.  \query{q0} and other
very simple queries are about 1.5\x{} faster.

\paragraph{Peak memory.}

Figure~\ref{fig:macrobenchmark}(b) shows peak memory consumption, as
reported as the operating system resident set size (RSS), for the
Flink or \dbsp processes.  In our case this is a single process; for
Flink, it is the sum of the RSS in the 8 task manager containers.  In
the measurements we did not include the cost incurred by the control
plane in either case (in Feldera's system, the pipeline manager; in
Flink, the job manager).

\dbsp uses between 0.03\x{} and 2.6\x{} as much memory of Flink, with a
geometric mean of 0.24\x{}.

\query{q0} and several other queries use 2~GiB or less memory with our
implementation, but over 17~GiB with Flink.  These queries are linear,
and do not require any state, or only minimal state, so \dbsp does not
allocate much memory.  Flink runs under the Java Virtual Machine,
which might cause it to allocate a high minimum amount of memory.

Most queries use less memory in our system than in Flink.  \query{q9}
and \query{q13} use substantially more, and \query{q18} and
\query{q19} use somewhat more, and storage reduces the RAM use
significantly.
