\section{Recursive queries}\label{sec:recursion}

Recursive queries are very useful in a many applications.
For example, graph algorithms such as graph reachability
or transitive closure are naturally expressed using recursive queries.

We introduce two simple \dbsp stream operators that are used for
expressing recursive query evaluation.  These operators allow us
to build circuits implementing looping constructs, which
are used to iterate computations until a fixed-point is reached.
The following definition allows us to describe what a fixed point is
in terms of streams:

\begin{definition}\label{def:zae}
We say that a stream $s \in \stream{A}$ is \defined{zero almost-everywhere} if it has a finite
number of non-zero values, i.e., there exists a time $t_0 \in \N$
s.t. $\forall t \geq t_0 . s[t] = 0$.
\noindent Denote the set of streams that are zero almost everywhere
by $\streamf{A} \subset \stream{A}$.
\end{definition}

\paragraph{Stream introduction}

The delta function (named from the Dirac delta function) $\delta : A \rightarrow \stream{A}$
produces a stream from a scalar value:
$$\delta(v)[t] \defn \left\{
\begin{array}{ll}
  v & \mbox{if } t = 0 \\
  0_A & \mbox{ otherwise}
\end{array}
\right.
$$

The input of $\delta$ has a single arrow, while the output has a
double arrow.  For example:

\begin{center}
\begin{tikzpicture}[auto,node distance=1cm,>=latex]
    \node[] (input) {\fbox{x}};
    \node[block, right of=input] (delta) {$\delta$};
    \node[right of=delta, node distance=2.2cm] (output) {\sv{x 0 0 0 0}};
    \draw[->] (input) -- (delta);
    \draw[->>] (delta) -- (output);
\end{tikzpicture}
\end{center}

\paragraph{Stream elimination}

We define the function $\int : \streamf{A} \rightarrow A$, over
streams that are zero almost everywhere, as $\int(s) \defn \sum_{t
  \geq 0} s[t]$.  $\int$ simply adds up all values in the input stream
and produces a scalar result with the sum.  $\int$ is closely related
to $\I$; if $\I$ is the indefinite (discrete) integral, $\int$ is the
definite (discrete) integral on the interval $0 - \infty$.  For
example, $\int(\sv{1 2 3 0 0}) = 6$.

In circuits constructed for many classes of queries, including
relational and Datalog queries given below, the $\int$ operator can be
``approximated'' without loss of precision by integrating until it
encounters the first 0.

Notice that the input of $\int$ is a double arrow, while the output is
a single arrow.  E.g.,:

\begin{center}
\begin{tikzpicture}[auto,node distance=1cm,>=latex]
    \node[] (input) {$\sv{1 2 3 0 0}$};
    \node[block, right of=input, node distance=2.2cm] (S) {$\int$};
    \node[right of=S] (output) {$\fbox{6}$};
    \draw[->>] (input) -- (S);
    \draw[->] (S) -- (output);
\end{tikzpicture}
\end{center}

%$\delta$ is the left inverse of $\int$, i.e.: $\int \circ \; \delta = \id_A$.
\begin{proposition}
$\delta$ and $\int$ are LTI.
\end{proposition}

\paragraph{Nested time domains}

So far we have used a tacit assumption that ``time'' is common for all
streams in a program.  For example, when we add two streams,
we assume that they use the same ``clock'' for the time dimension.
However, the $\delta$ operator creates a stream with a ``new'', independent time
dimension.  We require \emph{well-formed circuits}
to ``insulate'' such
nested time domains by ``bracketing'' them between a $\delta$
and an $\int$ operator:

\begin{center}
\begin{tikzpicture}[auto,node distance=1cm,>=latex]
    \node[] (input) {$i$};
    \node[block, right of=input] (delta) {$\delta$};
    \node[block, right of=delta] (f) {$Q$};
    \draw[->] (input) -- (delta);
    \draw[->>] (delta) -- (f);

    \node[block, right of=f] (S) {$\int$};
    \node[right of=S] (output) {$o$};
    \draw[->>] (f) -- (S);
    \draw[->] (S) -- (output);
\end{tikzpicture}
\end{center}

In the above circuit $Q$ is a streaming operator, but the entire
circuit is a scalar function, shown by the single input and output
arrows.

Algorithm~\ref{algorithm-rec} below, which translates recursive queries to
\dbsp circuits, always produces well-formed circuits.
%\begin{proposition}
%If $Q$ is time-invariant, the circuit above has the zero-preservation
%property: $\zpp{\int \circ\; Q \circ \delta_o}$.
%\end{proposition}

\subsection{Implementing recursive queries}\label{sec:datalog}

We describe the implementation of recursive queries in \dbsp for
stratified Datalog.
In general, a recursive Datalog program defines a set of
mutually recursive relations $O_1,..,O_n$ as an equation
$(O_1,..,O_n)=R(I_1,..,I_m, O_1,..,O_n)$, where $I_1,..,I_m$ are
input relations and $R$ is a non-recursive query.

We describe here the algorithm for the simpler case of a single-input,
single-output query\footnote{The general case in the companion
technical report~\cite{tr} is only slightly more involved.}.  The
input of our algorithm is a Datalog query of the form $O = R(I, O)$,
where $R$ is a relational, non-recursive query, producing a set as a
result, but whose output $O$ is also an input.  The output of the
algorithm is a \dbsp circuit which evaluates this recursive query
producing output $O$ when given the input $I$.  In this section we
build a non-incremental circuit, which evaluates the Datalog query; in
\refsec{sec:nested} we incrementalize this circuit.

\begin{algorithm}%[recursive queries]
  \label{algorithm-rec}
\noindent
\begin{enumerate}[nosep, leftmargin=\parindent]
\item Implement the non-recursive relational query $R$ as described in
    \secref{sec:relational} and Table~\ref{tab:relational}; this produces
    an acyclic circuit whose inputs and outputs are \zrs:
    \begin{center}
    \begin{tikzpicture}[auto,>=latex]
      \node[] (I) {\code{I}};
      \node[below of=I, node distance=.5cm] (O) {\code{O}};
      \node[block, right of=I] (R) {$R$};
      \node[right of=R] (o) {\code{O}};
      \draw[->] (I) -- (R);
      \draw[->] (O) -| (R);
      \draw[->] (R) -- (o);
    \end{tikzpicture}
    \end{center}

In all these diagrams we show input 0 of operator $R$ on the left, and
input 1 on the bottom.

\item Lift this circuit to operate on streams:
    \begin{center}
    \begin{tikzpicture}[auto,>=latex]
      \node[] (I) {\code{I}};
      \node[below of=I, node distance=.7cm] (O) {\code{O}};
      \node[block, right of=I] (R) {$\lift R$};
      \node[right of=R] (o) {$O$};
      \draw[->>] (I) -- (R);
      \draw[->>] (O) -| (R);
      \draw[->>] (R) -- (o);
    \end{tikzpicture}
    \end{center}

  We construct $\lift{R}$ by lifting each operator of the circuit individually
  according to Proposition~\ref{prop:distributivity}.

\item Build a cycle, connecting the output to the corresponding
recursive input via a delay:

 \begin{center}
\begin{tikzpicture}[auto,>=latex, node distance=.8cm]
  \node[] (I) {\code{I}};
  \node[block, right of=I, node distance=1cm] (R) {$\lift R$};
  \node[right of=R, node distance=1.5cm] (O) {\code{O}};
  \node[block, below of=R, node distance=.8cm] (z) {$\zm$};
  \draw[->>] (I) -- (R);
  \draw[->>] (R) -- node(o) {} (O);
  \draw[->>] (o) |- (z);
  \draw[->>] (z) -- (R);
 \end{tikzpicture}
 \end{center}

\item ``Bracket'' the circuit, first with $\I$ and $\D$ nodes, and
  then in $\delta$ and $\int$:

\begin{center}
\begin{tikzpicture}[auto,>=latex, node distance=.8cm]
  \node[] (Iinput) {\code{I}};
  \node[block, right of=Iinput] (ID) {$\delta$};
  \node[block, right of=ID] (II) {$\I$};
  \node[block, right of=II, node distance=1cm] (f) {$\lift{R}$};
  \node[block, right of=f, node distance=1.5cm] (D) {$\D$};
  \node[block, right of=D] (S) {$\int$};
  \node[right of=S] (output)  {\code{O}};
  \draw[->] (Iinput) -- (ID);
  \draw[->>] (ID) -- (II);
  \draw[->>] (II) -- (f);
  \draw[->>] (f) -- node (o) {$o$} (D);
  \draw[->>] (D) -- (S);
  \draw[->] (S) -- (output);
  \node[block, below of=f, node distance=.8cm] (z) {$\zm$};
  \draw[->>] (o) |- (z);
  \draw[->>] (z) -- (f);
\end{tikzpicture}
    \vspace{-2ex}
\end{center}
\end{enumerate}
\end{algorithm}

We argue that the cycle inside this circuit computes iteratively the fixed point of $R$.
The $\D$ operator yields the set of new Datalog facts (changes) computed by each iteration of the loop.
When the set of new facts becomes empty, the fixed point has been reached:

\begin{theorem}[Recursion correctness]\label{theorem:recursion}
If $\isset(\code{I})$, the output of the circuit above is
the relation $\code{O}$ as defined by the Datalog semantics of given program
$R$ as a function of the input relation \code{I}.
\end{theorem}
\label{proof-recursion}

\begin{proof}
Let us compute the contents of the $o$ stream, produced at the output
of $R$.  We show that this stream contains increasing approximations of the value of \code{O}.

Define the following one-argument function: $S(x) = \lambda x . R(\code{I}, x)$.
Notice that the left input of the $\lift{R}$ block is a constant stream
with the value \code{I}.  Due to the stratified nature of the language,
we must have $\ispositive(S)$, so $\forall x . S(x) \geq x$.
We get the following system of equations:
$$
\begin{aligned}
o[0] =& S(0) \\
o[t] =& S(o[t-1]) \\
\end{aligned}
$$
So, by induction on $t$ we have $o[t] = S^t(0)$, where by
$S^t$ we mean $\underbrace{S \circ S \circ \ldots \circ S}_{t}$.
$S$ is monotone; thus, if there is a time $k$ such that $S^k(0) = S^{k+1}(0)$, we have
$\forall j . S^{k+j}(0) = S^k(0)$.  Applying a $\D$ to this stream
will then produce a stream that is zero almost everywhere, and integrating
this result will return the last distinct value in the stream $o$.

This is essentially the definition of the semantics of a recursive Datalog relation:
$\code{O} = \fix{x}{R(\code{I}, x)}$.\qed
\end{proof}

Note that if the query $R$ computes over unbounded data domains (e.g.,
using integers with arithmetic), this construction does not guarantee
that at runtime a fixed point is reached.  But if a program does converge, the
above construction will find the least fixed point.

In fact, this circuit implements the standard \defined{na\"{\i}ve evaluation}
algorithm (e.g., see Algorithm~1 in \cite{greco-sldm15}).
Notice that the inner part of the circuit is the incremental
form of another circuit, since it is sandwiched between $\I$ and $\D$ operators.
Using the cycle rule of Proposition~\ref{prop-inc-properties} we can rewrite this circuit as:
%
\begin{center}
\begin{tikzpicture}[auto,>=latex]
  \node[] (Iinput) {$x$};
  \node[block, right of=Iinput] (Idelta) {$\delta$};
  \node[block, right of=Idelta, node distance=1.3cm] (f) {$\inc{(\lift{R})}$};
  \node[block, right of=f, node distance=1.5cm] (S) {$\int$};
  \node[right of=S] (output)  {$O$};
  \node[block, below of=f, node distance=.9cm] (z) {$\zm$};
  \draw[->] (Iinput) -- (Idelta);
  \draw[->] (f) -- node (o) {} (S);
  \draw[->>] (S) -- (output);
  \draw[->>] (o) |- (z);
  \draw[->>] (z) -- (f);
  \draw[->>] (Idelta) -- (f);
\end{tikzpicture}
\end{center}

This circuit implements \defined{semi-na\"{\i}ve evaluation}
(Algorithm~2 from~\cite{greco-sldm15}).  We have just proven the
correctness of semi-na\"{\i}ve evaluation as an immediate consequence
of the cycle rule!

%In \refsec{sec:recursive-example} we show a concrete example, applying Algorithm~\ref{algorithm-rec}
%to a recursive query computing the transitive closure of a graph.

\subsection{Example recursive query}\label{sec:recursive-example}

Let us apply the algorithm~\ref{algorithm-rec} to a concrete Datalog
program, which computes the transitive closure of a directed graph:

\begin{lstlisting}[language=ddlog,basicstyle=\small\ttfamily]
// Edge relation with head and tail
input relation E(h: Node, t: Node)
// Reach relation with source s and sink t
output relation R(s: Node, t: Node)
R(x, y) :- E(x, y).
R(x, y) :- E(x, z), R(z, y).
\end{lstlisting}

Step 1: we ignore the fact that R is both an input and an output and we implement
the \dbsp circuit corresponding to the body of the query.  This query could be expressed
in SQL as:

\begin{lstlisting}[language=SQL,basicstyle=\small\ttfamily]
( SELECT * FROM E )
UNION
( SELECT E.h , R.t
  FROM E JOIN R ON E.t = R.s )
\end{lstlisting}

\noindent Step 1:
This generates a circuit with inputs \code{E} and \code{R}:

\begin{tikzpicture}[>=latex, node distance=1.2cm]
  \node[] (E) {\code{E}};
  \node[above of=E, node distance=.6cm] (R1) {\code{R}};
  \node[block, right of=R1] (j) {$\bowtie_{t=s}$};
  \node[block, right of=j] (pj) {$\pi_{h, t}$};
  \node[block, circle, below of=pj, inner sep=0cm, node distance=.6cm] (plus) {$+$};
  \node[block, right of=plus] (d) {$\distinct$};
  \node[right of=d] (R) {\code{R}};
  \draw[->] (R1) -- (j);
  \draw[->] (E) -- (j);
  \draw[->] (j) -- (pj);
  \draw[->] (E) -- (plus);
  \draw[->] (pj) -- (plus);
  \draw[->] (plus) -- (d);
  \draw[->] (d) -- (R);
\end{tikzpicture}

\noindent Step 2: Lift the circuit by lifting each operator:

\begin{tikzpicture}[>=latex, node distance=1.4cm]
  \node[] (E) {\code{E}};
  \node[above of=E, node distance=.6cm] (R1) {\code{R}};
  \node[block, right of=R1] (j) {$\lift{\bowtie_{t=s}}$};
  \node[block, right of=j] (pj) {$\lift{\pi_{h, t}}$};
  \node[block, circle, below of=pj, node distance=.8cm, inner sep=0cm] (plus) {$+$};
  \node[block, right of=plus] (d) {$\lift{\distinct}$};
  \node[right of=d] (R) {\code{R}};
  \draw[->>] (R1) -- (j);
  \draw[->>] (E) -- (j);
  \draw[->>] (j) -- (pj);
  \draw[->>] (E) -- (plus);
  \draw[->>] (pj) -- (plus);
  \draw[->>] (plus) -- (d);
  \draw[->>] (d) -- (R);
\end{tikzpicture}

\noindent Step 3: Connect the feedback loop implied by \code{R}:

\begin{tikzpicture}[>=latex]
  \node[] (E) {\code{E}};
  \node[right of=E] (empty) {};
  \node[block, above of=empty, node distance=.6cm] (j) {$\lift{\bowtie_{t=s}}$};
  \node[block, right of=j, node distance=1.4cm] (pj) {$\lift{\pi_{h, t}}$};
  \node[block, circle, below of=pj, node distance=.8cm, inner sep=0cm] (plus) {$+$};
  \node[block, right of=plus, node distance=1.4cm] (d) {$\lift{\distinct}$};
  \draw[->>] (E) -- (j);
  \draw[->>] (j) -- (pj);
  \draw[->>] (E) -- (plus);
  \draw[->>] (pj) -- (plus);
  \draw[->>] (plus) -- (d);

  % generic part
  \node[right of=d, node distance=1.3cm] (output)  {\code{R}};
  \draw[->>] (d) -- node (o) {} (output);
  \node[block, above of=j, node distance=.8cm] (z) {$\zm$};
  \draw[->>] (o.center) |- (z);
  \draw[->>] (z) -- (j);
\end{tikzpicture}

\noindent Step 4: ``bracket'' the circuit with $\I$-$\D$, and with $\delta$-$\int$:

\noindent
\begin{tikzpicture}[>=latex, node distance=1.3cm]
  \node[] (Einput) {\code{E}};
  % generic part
  \node[block, right of=Einput, node distance=.8cm] (ID) {$\delta$};
  \node[block, right of=ID, node distance=.8cm] (E) {$\I$};

  % relational query
  \node[block, above of=E, node distance=.8cm] (j) {$\lift{\bowtie_{t=s}}$};
  \node[block, right of=j, node distance=1.4cm] (pj) {$\lift{\pi_{h, t}}$};
  \node[block, circle, below of=pj, node distance=.8cm, inner sep=0cm] (plus) {$+$};
  \node[block, right of=plus] (d) {$\lift{\distinct}$};
  \draw[->>] (E) -- (j);
  \draw[->>] (j) -- (pj);
  \draw[->>] (E) -- (plus);
  \draw[->>] (pj) -- (plus);
  \draw[->>] (plus) -- (d);

  % generic part
  \node[block, right of=d, node distance=1.3cm] (D) {$\D$};
  \node[block, right of=D, node distance=1cm] (S) {$\int$};
  \node[right of=S, node distance=.8cm] (output)  {\code{R}};
  \draw[->] (Einput) -- (ID);
  \draw[->>] (ID) -- (E);
  \draw[->>] (d) -- node (o) {} (D);
  \draw[->>] (D) -- (S);
  \draw[->] (S) -- (output);
  \node[block, above of=j, node distance=.8cm] (z) {$\zm$};
  \draw[->>] (o.center) |- (z);
  \draw[->>] (z) -- (j);
\end{tikzpicture}

The above circuit is a complete implementation of the non-streaming
recursive query; given an input relation \code{E} it will produce
its transitive closure \code{R} as output.

Now we use semina\"ive evaluation to rewrite the circuit (to save
space we omit the indices from $\pi$ and $\sigma$):

\noindent
\begin{tikzpicture}[>=latex, node distance=1.4cm]
  \node[] (Einput) {\code{E}};
  % generic part
  \node[block, right of=Einput, node distance=.8cm] (E) {$\delta$};

  % relational query
  \node[block, above of=E, node distance=.8cm] (j) {$\inc{(\lift{\bowtie})}$};
  \node[block, right of=j, node distance=1.5cm] (pj) {$\inc{(\lift{\pi})}$};
  \node[block, circle, below of=pj, node distance=.8cm, inner sep=0cm] (plus) {$+$};
  \node[block, right of=plus] (d) {$\inc{(\lift{\distinct})}$};
  \draw[->>] (E) -- (j);
  \draw[->>] (j) -- (pj);
  \draw[->>] (E) -- (plus);
  \draw[->>] (pj) -- (plus);
  \draw[->>] (plus) -- (d);

  % generic part
  \node[block, right of=d, node distance=1.5cm] (S) {$\int$};
  \node[right of=S, node distance=.8cm] (output)  {\code{R}};
  \draw[->] (Einput) -- (E);
  \draw[->>] (d) -- node (o) {} (S);
  \draw[->] (S) -- (output);
  \node[block, above of=j, node distance=.9cm] (z) {$\zm$};
  \draw[->>] (o.center) |- (z);
  \draw[->>] (z) -- (j);
\end{tikzpicture}

Using the linearity of $\lift\pi$, this can be rewritten as:

\noindent
\begin{tikzpicture}[>=latex, node distance=1.5cm]
  \node[] (Einput) {\code{E}};
  % generic part
  \node[block, right of=Einput, node distance=.8cm] (E) {$\delta$};

  % relational query
  \node[block, above of=E, node distance=.8cm] (j) {$\inc{(\lift{\bowtie})}$};
  \node[block, right of=j] (pj) {$\lift{\pi}$};
  \node[block, circle, below of=pj, node distance=.8cm, inner sep=0cm] (plus) {$+$};
  \node[block, right of=plus] (d) {$\inc{(\lift{\distinct})}$};
  \draw[->>] (E) -- (j);
  \draw[->>] (j) -- (pj);
  \draw[->>] (E) -- (plus);
  \draw[->>] (pj) -- (plus);
  \draw[->>] (plus) -- (d);

  % generic part
  \node[block, right of=d] (S) {$\int$};
  \node[right of=S, node distance=.8cm] (output)  {\code{R}};
  \draw[->] (Einput) -- (E);
  \draw[->>] (d) -- node (o) {} (S);
  \draw[->] (S) -- (output);
  \node[block, above of=j, node distance=.8cm] (z) {$\zm$};
  \draw[->>] (o.center) |- (z);
  \draw[->>] (z) -- (j);
\end{tikzpicture}


