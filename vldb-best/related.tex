\section{Related work}\label{sec:related}

Incremental view
maintenance~\cite{buneman-actd79,blakeley-sigmod86,gupta-sigmod93,chaudhuri-icde95,gupta-idb95,chirkova-book12}
is a much studied problem in databases.  A survey of results for
Datalog queries is present in~\cite{motik-ai19}.  The standard
approach is as follows: given a query $Q$, discover a ``delta query'',
a ``differential'' version $\Delta Q$ that satisfies the equation:
$Q(d+\Delta d)=Q(d)+\Delta Q(d,\Delta d)$, and which can be used to
compute the change for a new input reusing the previous output.
DBToaster introduced recursive recursive
IVM~\cite{ahmad-vldb09,koch-pods10,nikolic-sigmod16}, where the
incrementalization process is repeated for the delta query.  Our
definition of IVM is subtly different from the above one, as IVM is
defined as a stream computation, well-defined for any query.

\cite{bello-vldb98} describes an early implementation in Oracle~8,
which handles a limited set of queries.  Many custom algorithms were
published for various classes of queries:
e.g. \cite{griffin-sigmod98,larson-icde07} for various classes of
joins, \cite{koch-pods16} for positive nested relational calculus,
\cite{gupta-infsys06} for relational and aggregate operators;
\cite{kara-tds20} is optimized for triangle queries;
DYN~\cite{idris-sigmod17,idris-vldb18,idris-sigmod19} focuses on
acyclic conjunctive queries: instead of keeping the output view
materialized they build data structures that allow efficiently
querying the output views.  PAI maps~\cite{abeysinghe-sigmod22} are
specially designed for queries with correlated aggregations.
\cite{palpanas-vldb02} discusses non-distributive aggregate functions.
\cite{katsis-sigmod15} uses primary key information to compress the
representation of the deltas, and using an ``update'' operator that is
similar to our ``upsert'' operator.  AJU~\cite{wang-sigmod20} and
\cite{svingos-amd23} focus on using foreign key information to
optimize query plan generation.  These techniques are only sound in
the absence of deletions and updates; our implementation uses these
optimizations as well.  Some algorithms apply to sets, some work for
multisets~\cite{griffin-sigmod95}.  Many of these formalisms look very
complicated because they deal with ``insertions'', ``deletions'', and
``update'' changes separately.  \zrs are a much more compact tool for
describing such algorithms.  In some sense the \dbsp theory, through
the chain rule, enables us to reuse all of these results (and any
future schemes designed for particular classes of subqueries): given a
good implementation strategy for a particular query plan it can be
reused as a subplan in any query which can be made to use that
particular plan.

\dbsp as described implies an ``eager'' execution model: it constantly
maintains the entire contents of any number of views, even if no one
really wants to inspect the views.  In contrast, ``lazy''
models~\cite{hanson-sigmod87} only build part of the views when the
views are inspected.  Such models have the potential to be more
efficient.  A simple way to implement a ``lazy'' model using \dbsp is
to essentially accumulate all input changes as \zrs and apply the
incremental algorithm only when the output view is queried.  Between
``lazy'' and ``eager'' one can place ``snapshot'' views, which are
updated periodically~\cite{colby-sigmod97}.  Snowflake offers all
these models~\cite{akidau-amd23}.

\dbsp is a bottom-up system, which always produces eagerly the
\emph{changes} to the output views.  Instead of maintaining the output
view entirely, \dbsp proposes generating deltas as the output of the
computation (similar to the kSQL~\cite{jafarpour-edbt19} \texttt{EMIT
  CHANGES} queries).  The idea that both inputs and outputs to an IVM
system are streams of changes seems trivial, but this is key to the
symmetry of our solution: both in our definition of
IVM~(\ref{def:inc}), and the fundamental reason that the chain rule
exists --- the chain rule is the one that makes our structural
induction IVM algorithm possible.

Several IVM algorithms for Datalog-like languages use counting based
approaches~\cite{Dewan-iis92,motik-aaai15} that maintain the number of
derivations of each output fact: DRed~\cite{gupta-sigmod93} and its
variants~\cite{Ceri-VLDB91,Wolfson-sigmod91,Staudt-vldb96,Kotowski-rr11,Lu-sigmod95,Apt-sigmod87},
the backward-for\-ward algorithm and
variants~\cite{motik-aaai15,Harrison-wdd92,motik-ai19}.  \dbsp is more
general, and our incrementalization algorithm handles arbitrary
recursive queries and generates more efficient plans for recursive
queries in the presence of arbitrary updates (especially deletions,
where competing approaches may over-delete).  Interestingly, the \zrs
weights in \dbsp are related to the counting-number-of-derivations
approaches, but our use of the $\distinct$ operator shows that precise
counting is not necessary.

Picallo et al.~\cite{picallo-scop19} provide a general solution to IVM
for rich languages.  Unlike their proposal, \dbsp requires a group
structure on the values operated on; this assumption has two major
practical benefits: it simplifies the mathematics considerably (e.g.,
Picallo uses monoid actions to model changes), and it provides a
general, simple algorithm for incrementalizing arbitrary programs.
The downside of \dbsp is that one has to find a suitable group
structure (e.g., \zrs for sets) to ``embed'' the computation.
Picallo's notion of ``derivative'' is not unique: they need creativity
to choose the right derivative definition, we need creativity to find
the right group structure.

Finding a suitable group structure has proven easy for relations
(both~\cite{koch-pods10} and~\cite{green-tcs11} use \zrs to uniformly
model data and insertions/deletions), but it is not obvious how to do
it for other data types, such as sorted collections, or tree-shaped
collections (e.g., XML or JSON documents)~\cite{foster-planx08}.  An
intriguing question is ``what other interesting group structures could
this be applied to besides \zrs?''  Papers such
as~\cite{nikolic-icmd18} explore other possibilities, such as matrix
algebra, linear ML models, or conjunctive queries.

\dbsp can also model window and stream database
queries~\cite{arasu-tr02,aurora} such as CQL queries.
\cite{begoli-sigmod19} proposes using SQL to express both standard
database queries and streaming queries; it also proposes some
extensions to SQL specific to streaming systems.  The \dbsp theory
allows us to more precisely understand the classes of queries that
cannot be expressed in SQL.  A SQL query is a function of the state of
the database; in other words, a SQL query cannot provide different
results based on the order of insertions of tuples in a table.
Streaming systems however can.  \dbsp also enables us to handle
generalizations of the proposed concepts -- for example, we believe
that the ``timestamps'' attached by streaming systems to ``events'' do
not need any special treatment.

\cite{bonifati-iclp2018} implemented a verified IVM algorithm for a
particular class of graph queries called Regular Datalog, with an
implementation machine-checked in the Coq proof assistant. Their focus
is on a particular algorithm and the approach does not consider other
SQL operators, general recursion, or custom operators (although it is
modular in the sense that it works on any query by incrementalizing it
recursively). Furthermore, for all queries a deletion in the input
change stream requires running the non-incremental query to recover.
We formally verify the theorems in our paper, which are much broader
in scope, but not our implementations.

\dbsp is also related to Differential Dataflow
(DD)~\cite{mcsherry-cidr13,murray-sosp13,chothia-vldb16} and its
theoretical foundations~\cite{abadi-fossacs15}.  DD's computational
model is more powerful than \dbsp, since it models time values as part
of an arbitrary lattice.  In fact, DD is the only other framework
which we are aware of that can incrementalize recursive queries as
efficiently as \dbsp does.  In contrast, our model uses either
``linear'' times, or nested time dimensions via the modular lifting
transformer ($\lift{}$).  \dbsp can express both incremental and
non-incremental computations.  Most importantly, \dbsp comes with
Algorithm~\ref{algorithm-inc}, a syntax-directed translation that can
convert any expressible query into an incremental version --- in DD
users have to assemble incremental queries manually using incremental
operators.  materialize.com offers a product that automates
incrementalization for SQL queries based on DD.  The Differential
Datalog~\cite{ryzhyk-datalog19} project compiles Datalog to DD.
Unlike DD, \dbsp is a modular theory, which easily accommodates the
addition of new operators: as long as we can express a new operator as
a \dbsp circuit, we can (1) define its incremental version, (2) apply
the incrementalization algorithm to obtain an efficient incremental
implementation, and (3) be confident that it composes with any other
operators.

Many custom streaming systems have been implemented:
Storm~\cite{toshnival-sigmod14}, Spark
Streaming~\cite{zaharia-sosp13}, Flink~\cite{carbone-ieee15},
Samza~\cite{noghabi-vldb17}, Beam~\cite{akidau-vldb15}, Kafka
Streams~\cite{wang-sigmod21}, and many have adopted SQL dialectsL
Spark Structured Streaming~\cite{armbrust-sigmod18}, Spark SQL,
KSQL~\cite{jafarpour-edbt19}.  These sytems usually sacrifice some of
the nice properties of database systems in order to compute
efficiently over unbounded streams.  These systems may only support
restricted classes of queries.  We believe that in the future
databases will incorporate the best features of streaming systems, and
that \dbsp shows one way this can be achieved.

\cite{akidau-amd23,akidau-debs24} describe the Snowflake incremental
and streaming capabilities.  In Showflake ``streams'' are database
table that store the history of changes to a table.  Dynamic tables
are views which are periodically refreshed, at user-specified
intervals.  These can be updated either incrementally or using batch
recomputation; the system chooses a stragegy based on the refres
period.  The views provide snapshot isolation, which is similar to the
DBSP consistency model.

The \code{pg\_ivm}~\cite{pgivm} project ofers an open-source Postgres
module which adds IVM capabilities.  The supported set of queries has
some significant restrictions.

The \dbsp model is simple enough so it can be implemented in a few
hundred lines of Python~\cite{dbsp-python}.
