\section{IVM for the Relational Algebra}\label{sec:relational}

Results in \secref{sec:streams} and~\secref{sec:incremental}
apply to streams of arbitrary group values.  In this
section we apply these results to
IVM for relational databases.  As explained in the introduction, our goal is to
efficiently compute the incremental version of a relational query $Q$
defining a view.

However, we face a technical problem: the $\I$ and $\D$ operators were
defined on Abelian groups, and relational databases in general are
not Abelian groups, since they operate on sets.  Fortunately,
there is a well-known tool in the database literature
which converts set operations into group operations by using \zrs
(also called z-relations~\cite{green-tcs11}) to represent sets.

We start by defining the \zrs group, and then we review how relational
queries are converted into \dbsp circuits over \zrs.  We show
in~\refsec{sec:relational-operators} that this translation is
efficiently incrementalizable because many primitive relational
operations use LTI \zr operators.

\vspace{-3ex}

\subsection{\zrs as an Abelian group}

A \zr is a database table where each row has an associated weight,
possibly negative.  Given a set $A$, we define \defined{\zrs} over $A$
as functions with \emph{finite support} from $A$ to $\Z$.  These are
functions $f: A \rightarrow \Z$ where $f(x) \not= 0$ for at most a
finite number of values $x \in A$.  We also write $\Z[A]$ for the type
of \zrs with elements from $A$.  Values in $\Z[A]$ can be thought of
as key-value maps with keys in $A$ and values in $\Z$, justifying the
array indexing notation.  If $m \in \Z[A]$ we write $m[a]$ instead of
$m(a)$, again using an indexing notation.

A particular \zr $m \in \Z[A]$ can be denoted by enumerating its
elements that have non-zero weights and their corresponding weights:
$m = \{ x_1 \mapsto w_1, \dots, x_n \mapsto w_n \}$.
We call $w_i \in \Z$ the \defined{weight}
of $x_i \in A$.  Weights can be negative.
We say that $x \in m$ iff $m[x] \not= 0$.

The following table shows an example \zr with three rows, let's call
it $R$.  The first row has value \texttt{joe} and weight 1.  We never
show rows with weight 0.
%
\begin{center}
\begin{tabular}{|c|c|}\hline
  Row & Weight \\ \hline
  joe & 1 \\
  mary & 2 \\
  anne & -1 \\ \hline
\end{tabular}
\end{center}

$R$ has three elements in its domain, \texttt{joe} with weight 1 (so
$R[\texttt{joe}] = 1$), \texttt{mary} with weight 2, and \texttt{anne}
with weight $-1$.  So \texttt{joe} $\in R$, \texttt{mary} $\in R$,
and \texttt{anne} $\in R$.

Since $\Z$ is an Abelian ring, $\Z[A]$ is also an Abelian ring, and
thus a group $(\Z[A], +_{\Z[A]}, 0_{\Z[A]}, -_{\Z{A}})$.  Addition and
subtraction are defined pointwise: $\forall x \in A . \\ (f +_{\Z[A]}
g)(x) = f(x) + g(x) .$ The $0$ element of $\Z[A]$ is the function
$0_{\Z[A]}$ defined by $0_{\Z[A]}(x) = 0 .  \forall x \in A$.  For
example, $R + R = \{ \texttt{joe} \mapsto 2, \texttt{mary} \mapsto 4,
\texttt{anne} \mapsto -2 \}$.  Since \zrs form a group, all results
from \secref{sec:streams} apply to streams over \zrs.

\zrs generalize sets and bags.  A set with elements from $A$ can be
represented as a \zr by associating a weight of 1 with each element.
Bags are \zrs where all weights are positive.  Crucially, \zrs can
also represent \emph{changes} to sets and bags.  Negative weights in a
change represent elements that are being ``removed''.

The remaining definitions in this section will be used to argue that
circuits based on \zrs can exactly implement the relational algebra
operators.

\begin{definition}
We say that a \zr represents a \defined{set} if the weight of every
element is one.  We define a function to check this property
$\isset : \Z[A] \rightarrow \B$\index{isset}
given by:
$$\isset(m) \defn \left\{
\begin{array}{ll}
  \mbox{true} & \mbox{ if } m[x] = 1, \forall x \in m \\
  \mbox{false} & \mbox{ otherwise}
\end{array}
\right.
$$
\end{definition}

\ifzsetexamples
\noindent For our example $\isset(R) = \mbox{false}$, since $R[\texttt{anne}] = -1$.
\fi

\begin{definition}
We say that a \zr is \defined{positive} (or a \defined{bag}) if the weight of every element is
positive.  We define a function to check this property \\
$\ispositive : \Z[A] \rightarrow \B$\index{ispositive}
given by
$$\ispositive(m) \defn \left\{
\begin{array}{ll}
  \mbox{true} & \mbox{ if } m[x] \geq 0, \forall x \in A \\
  \mbox{false} & \mbox{ otherwise}
\end{array}
\right.$$
\end{definition}
If $\isset(m)$, then $\ispositive(m)$.  For our example \zr, \\
$\ispositive(R) = \mbox{false}$.

We write $m \geq 0$ when $m$ is positive.  For positive $m, n \in
\Z[A]$ we write $m \geq n$ for iff $m - n \geq 0$.  $\geq$ is a
partial order.

We call a function $f : \Z[A] \rightarrow \Z[B]$ \defined{positive} if it maps
positive values to positive values:
$\forall x \in \Z[A]~.~x \geq 0_{\Z[A]} \Rightarrow f(x) \geq 0_{\Z[B]}$.
We use the same notation for functions: $\ispositive(f)$.

\begin{definition}[distinct]
The function $\distinct: \\ \Z[A] \rightarrow \Z[A]$\index{distinct}
``converts'' a \zr into a set:
$$\distinct(m)[x] \defn \left\{
\begin{array}{ll}
  1 & \mbox{ if } m[x] > 0 \\
  0 & \mbox{ otherwise}
\end{array}
\right.
$$
\end{definition}

Notice that $\distinct$ ``removes'' duplicates from multisets, and it also eliminates
elements with negative weights.
\ifzsetexamples
$\distinct(R) = \{ \texttt{joe} \mapsto 1, \texttt{mary} \mapsto 1 \}$.
\fi
While very simple, this definition of $\distinct$ has been carefully
chosen to enable us to implement the relational (set) operators
using \zrs operators.
%Circuits derived from relational queries only compute on positive \zrs.

%\begin{definition}(mononotonicity)
%A stream $s \in \stream{\Z[A]}$ is \defined{positive} if every value of the stream is positive:
%$s[t] \geq 0 . \forall t \in \N$.
%A stream $s \in \stream{\Z[A]}$ is \defined{monotone} if $s[t] \geq s[t-1], \forall t \in \N$.
%\end{definition}
%
%If $s \in \stream{\Z[A]}$ is positive, then $\I(s)$ is monotone.
%If $s \in \stream{\Z[A]}$ is monotone, $\D(s)$ is positive.
%
\subsection{Relational operators on \zrs}\label{sec:relational-operators}

The fact that relational algebra can be implemented by computations
on \zrs has been shown before, e.g.~\cite{green-pods07}.  The translation
of the relational operators to \dbsp is shown in Table~\ref{tab:relational}.
The first row of the table shows that a composite query is translated
recursively.  This gives us a recipe for
translating an arbitrary relational query plan into a \dbsp circuit.

\newlength{\commentsize}
\setlength{\commentsize}{7cm}
\begin{table*}[h]
\centering
\small
\caption{Implementation of SQL relational set operators as circuits
  computing on \zrs.\label{tab:relational}}
\begin{tabular}{|m{1.4cm}m{3.6cm}m{3.5cm}m{\commentsize}|} \hline
Operation & SQL example & \dbsp circuit & Details \\ \hline
Composition &
 \begin{lstlisting}[language=SQL]
SELECT ... AS O FROM
(SELECT ... AS I
 FROM I0)
\end{lstlisting}
 &
 \begin{tikzpicture}[auto,>=latex]
  \node[] (I) {\code{I0}};
  \node[block, right of=I] (CI) {$C_I$};
  \draw[->] (I) -- (CI);
  \node[block, right of=CI] (CO) {$C_O$};
  \node[right of=CO] (O) {\code{O}};
  \draw[->] (CI) -- (CO);
  \draw[->] (CO) -- (O);
\end{tikzpicture}
 &
 \parbox[b][][t]{\commentsize}{
  $C_I$ circuit for inner query, \\
   $C_O$ circuit for outer query.}
 \\ \hline
Union &
\begin{lstlisting}[language=SQL]
(SELECT * FROM I1)
UNION
(SELECT * FROM I2)
\end{lstlisting}
&
\begin{tikzpicture}[auto,>=latex]
  \node[] (input1) {\code{I1}};
  \node[below of=input1, node distance=.4cm] (midway) {};
  \node[below of=midway, node distance=.4cm] (input2) {\code{I2}};
  \node[block, shape=circle, right of=midway, inner sep=0in] (plus) {$+$};
  \node[block, right of=plus] (distinct) {$\distinct$};
  \node[right of=distinct] (output) {\code{O}};
  \draw[->] (input1) -| (plus);
  \draw[->] (input2) -| (plus);
  \draw[->] (plus) -- (distinct);
  \draw[->] (distinct) -- (output);
\end{tikzpicture}
& $\distinct$ eliminates duplicates.  An implementation of
\texttt{UNION ALL} does not need the $\distinct$.
\\ \hline
Projection &
\begin{lstlisting}[language=SQL]
SELECT DISTINCT I.c
FROM I
\end{lstlisting}
&
\begin{tikzpicture}[auto,>=latex]
  \node[] (input) {\code{I}};
  \node[block, right of=input] (pi) {$\pi_c$};
  \node[block, right of=pi, node distance=1.2cm] (distinct) {$\distinct$};
  \node[right of=distinct] (output) {\code{O}};
  \draw[->] (input) -- (pi);
  \draw[->] (pi) -- (distinct);
  \draw[->] (distinct) -- (output);
\end{tikzpicture}
&
\parbox[b][][t]{\commentsize}{
  Project each row with its weight unchanged.
  Add up weights of identical rows.
}
\\ \hline
Filtering &
\begin{lstlisting}[language=SQL]
SELECT * FROM I
WHERE P(...)
\end{lstlisting}
&
\begin{tikzpicture}[auto,>=latex]
  \node[] (input) {\code{I}};
  \node[block, right of=input] (map) {$\sigma_P$};
  \node[right of=map] (output) {\code{O}};
  \draw[->] (input) -- (map);
  \draw[->] (map) -- (output);
\end{tikzpicture}
&
\parbox[b][][t]{\commentsize}{
  P is a predicate applied to each row.
  Select each row separately.  If the row is selected, preserve the
  weight, else make the weight 0.
}
\\ \hline
\parbox[b][][t]{1cm}{
Cartesian \\
product} &
\begin{lstlisting}[language=SQL]
SELECT I1.*, I2.*
FROM I1, I2
\end{lstlisting}
&
\begin{tikzpicture}[auto,>=latex]
  \node[] (i1) {\code{I1}};
  \node[below of=i1, node distance=.4cm] (midway) {};
  \node[below of=midway, node distance=.4cm] (i2) {\code{I2}};
  \node[block, right of=midway] (prod) {$\times$};
  \node[right of=prod] (output) {\code{O}};
  \draw[->] (i1) -| (prod);
  \draw[->] (i2) -| (prod);
  \draw[->] (prod) -- (output);
\end{tikzpicture}
&
\parbox[b][][t]{\commentsize}{
  The weight of the pair (a,b) is the product of the the weights of a
  and b.
}
\\ \hline
Equi-join &
\begin{lstlisting}[language=SQL]
SELECT I1.*, I2.*
FROM I1 JOIN I2
ON I1.c1 = I2.c2
\end{lstlisting}
&
\begin{tikzpicture}[auto,>=latex]
  \node[] (i1) {\code{I1}};
  \node[below of=i1, node distance=.4cm] (midway) {};
  \node[below of=midway, node distance=.4cm] (i2) {\code{I2}};
  \node[block, right of=midway] (prod) {$\bowtie_{c1 = c2}$};
  \node[right of=prod] (output) {\code{O}};
  \draw[->] (i1) -| (prod);
  \draw[->] (i2) -| (prod);
  \draw[->] (prod) -- (output);
\end{tikzpicture}
&
\parbox[b][][t]{\commentsize}{
  Multiply the weights of the rows that appear in the output.
}
\\ \hline
Intersection &
\begin{lstlisting}[language=SQL]
(SELECT * FROM I1)
INTERSECT
(SELECT * FROM I2)
\end{lstlisting}
&
\begin{tikzpicture}[auto,>=latex]
  \node[] (i1) {\code{I1}};
  \node[below of=i1, node distance=.4cm] (midway) {};
  \node[below of=midway, node distance=.4cm] (i2) {\code{I2}};
  \node[block, right of=midway] (prod) {$\bowtie$};
  \node[right of=prod] (output) {\code{O}};
  \draw[->] (i1) -| (prod);
  \draw[->] (i2) -| (prod);
  \draw[->] (prod) -- (output);
\end{tikzpicture}
&
Special case of equi-join when both relations have the same schema.
\\ \hline
Difference &
\begin{lstlisting}[language=SQL]
SELECT * FROM I1
EXCEPT
SELECT * FROM I2
\end{lstlisting}
&
\begin{tikzpicture}[auto,>=latex, node distance=.7cm]
  \node[] (i1) {\code{I1}};
  \node[below of=i1, node distance=.4cm] (midway) {};
  \node[below of=midway, node distance=.4cm] (i2) {\code{I2}};
  \node[block, shape=circle, inner sep=0in, right of=i2] (m) {$-$};
  \node[block, right of=midway, shape=circle, inner sep=0in, node distance=1.3cm] (plus) {$+$};
  \node[block, right of=plus, node distance=1cm] (distinct) {$\distinct$};
  \node[right of=distinct, node distance=1cm] (output) {\code{O}};
  \draw[->] (i1) -| (plus);
  \draw[->] (i2) -- (m);
  \draw[->] (m) -| (plus);
  \draw[->] (plus) -- (distinct);
  \draw[->] (distinct) -- (output);
\end{tikzpicture}
&
$\distinct$ removes rows with negative weights from the result.
\\ \hline
\end{tabular}
\end{table*}

The translation is fairly straightforward, but many operators require
the application of a $\distinct$ \textbf{to produce sets}.
For example, $a \cup b = \distinct(a + b)$, $a \setminus b =
\distinct(a - b)$, $(a \times b)((x,y)) = a[x] \times b[y]$.
%\paragraph{Correctness of the \dbsp implementations}\label{sec:correctness}
%
%A relational query $Q$ that transforms
%a set $V$ into a set $U$ is implemented by a \dbsp computation $Q'$ on
%\zrs.  The correctness of the implementation requires the following
%diagram to commute:
%
%\begin{center}
%\begin{tikzpicture}
%  \node[] (V) {$V$};
%  \node[below of=V] (VZ) {$VZ$};
%  \node[right of=V, node distance=2cm] (U) {$U$};
%  \node[below of=U] (UZ) {$UZ$};
%  \draw[->] (V) -- node (f) [below] {$Q$} (U);
%  \draw[->] (V) --  node (s) [left] {tozset}(VZ);
%  \draw[->] (VZ) -- node (f) [above] {$Q'$} (UZ);
%  \draw[->] (UZ) -- node (d) [right] {toset} (U);
%\end{tikzpicture}
%\end{center}
%
%(The correctness of
%this implementation is predicated on $Q'$'s inputs being
%sets, an invariant which needs to be maintained by the environment.)
%The ``$\mbox{toset}$'' and ``$\mbox{tozset}$'' functions convert sets to \zrs and
%vice-versa, in the expected way:
%
%$\mbox{toset}: \Z[A] \to 2^A$ is defined as $\mbox{toset}(m) \defn \cup_{x \in \distinct(m)} \{ x \}$.
%
%$\mbox{tozset}: 2^A \to \Z[A]$ is defined as $\mbox{tozset}(s) \defn \sum_{x \in s} 1 \cdot x$.
%
%All standard algebraic properties
%of the relational operators can be used to optimize circuits
%(they can even be applied to queries before building the circuits).
%
Notice that the use of the $\distinct$ operator allows \dbsp to model
the \emph{full relational algebra}, including set difference (and not
just the positive fragment).

Prior work (e.g., Proposition 6.13 in~\cite{green-tcs11}) has shown
how some invocations of $\distinct$ can be eliminated from query plans
without changing the query semantics; we will see that the incremental
version of $\distinct$ incurs significant space costs, so it is worth
minimizing its use.

\begin{proposition}\label{prop-distinct-delay}
Let $Q$ be one of the following \zrs operators: filtering $\sigma$,
join $\bowtie$, or Cartesian product $\times$.  Then we have $\forall
i \in \Z[A]~.~\ispositive(i) \Rightarrow Q(\distinct(i)) =
\distinct(Q(i))$.  (For binary operators, on the left of $=$ the
$\distinct$ is applied to every input.)
\end{proposition}

\noindent
\begin{tabular}{m{3.5cm}m{.1cm}m{3.5cm}}
\begin{tikzpicture}[auto,>=latex]
  \node[] (input) {$i$};
  \node[block, right of=input, node distance=1.1cm] (distinct) {$\distinct$};
  \node[block, right of=distinct, node distance=1.2cm] (q) {$Q$};
  \node[right of=q] (output)  {$o$};
  \draw[->] (input) -- (distinct);
  \draw[->] (distinct) -- (q);
  \draw[->] (q) -- (output);
\end{tikzpicture}
&
$\cong$
&
\begin{tikzpicture}[auto,>=latex]
  \node[] (input) {$i$};
  \node[block, right of=input] (q) {$Q$};
  \node[block, right of=q, node distance=1.2cm] (distinct1) {$\distinct$};
  \node[right of=distinct1, node distance=1.2cm] (output)  {$o$};
  \draw[->] (input) -- (q);
  \draw[->] (q) -- (distinct1);
  \draw[->] (distinct1) -- (output);
\end{tikzpicture}
\end{tabular}

This rule allows us to delay the application of $\distinct$.

\begin{proposition}\label{prop-distinct-once}
Let $Q$ be one of the following \zrs operators: filtering $\sigma$,
projection $\pi$, map($f$)\footnote{Technically, map (applying a user-defined
function to each row) is not relational.},
addition $+$, join $\bowtie$, or
Cartesian product $\times$.
Then we have \\
$\ispositive(i) \Rightarrow \distinct(Q(\distinct(i))) = \distinct(Q(i))$.
\end{proposition}

\noindent
\begin{tabular}{m{6.5cm}m{.5cm}}
\begin{tikzpicture}[auto,>=latex]
  \node[] (input) {$i$};
  \node[block, right of=input, node distance=1.5cm] (distinct) {$\distinct$};
  \node[block, right of=distinct, node distance=1.5cm] (q) {$Q$};
  \node[block, right of=q, node distance=1.5cm] (distinct1) {$\distinct$};
  \node[right of=distinct1, node distance=1.5cm] (output)  {$o$};
  \draw[->] (input) -- (distinct);
  \draw[->] (distinct) -- (q);
  \draw[->] (q) -- (distinct1);
  \draw[->] (distinct1) -- (output);
\end{tikzpicture}
&
$\cong$ \\
\begin{tikzpicture}[auto,>=latex]
  \node[] (input) {$i$};
  \node[block, right of=input] (q) {$Q$};
  \node[block, right of=q, node distance=1.5cm] (distinct1) {$\distinct$};
  \node[right of=distinct1, node distance=1.5cm] (output)  {$o$};
  \draw[->] (input) -- (q);
  \draw[->] (q) -- (distinct1);
  \draw[->] (distinct1) -- (output);
\end{tikzpicture}
\end{tabular}

These properties allow us to ``consolidate'' distinct operators by
performing one $\distinct$ at the end of a chain of computations.
This optimization is also used in traditional database optimizers.

\vspace{-3ex}
\subsection{Incremental view maintenance}\label{sec:ivm}

Let us consider a relational query $Q$ defining a view $V$.  To create
a circuit that maintains incrementally $V$ we apply the following
mechanical steps:

\begin{algorithm}[incremental view maintenance]
  \label{algorithm-inc}\quad
\begin{enumerate}[nosep, leftmargin=\parindent]
    \item Translate $Q$ into a circuit using the rules in Table~\ref{tab:relational}.
    \item Apply $\distinct$ elimination rules
      (Propositions~\ref{prop-distinct-delay},
      \ref{prop-distinct-once})\footnote{If the rules are applied
      until convergence, the order in which the rules
      are applied does not matter, since the algorithm is confluent:
      it always produces the same final result.}.
    \item Lift the whole circuit, by applying Proposition~\ref{prop:distributivity},
    converting it to a circuit operating on streams.
    \item Incrementalize the whole circuit ``surrounding'' it with $\I$ and $\D$.
    \item Apply the chain rule from
      Proposition~\ref{prop-inc-properties} on the circuit to optimize
      the implementation.
\end{enumerate}
\end{algorithm}

This algorithm is deterministic and its running time
is given by the number of operators in the query.
Step (2) generates an equivalent circuit, with possibly fewer
$\distinct$ operators.  Step (3) yields a circuit that consumes a
\emph{stream} of complete database snapshots and outputs a stream of
complete view snapshots. Step (4) yields a circuit that consumes a
stream of \emph{database changes} and outputs a stream of \emph{view
changes}; however, the internal operation of the circuit is
non-incremental, as it rebuilds the complete database using
integration operators.  Step (5) incrementalizes the circuit by
replacing each primitive operator with its incremental version.

Most of the operators that appear in the circuits in
Table~\ref{tab:relational} are linear, and thus have very efficient
incremental versions (we discuss complexity in
\refsec{sec:complexity}).  A notable exception is $\distinct$.  The
next proposition shows that the incremental version of $\distinct$ is
also efficient, and it can be computed by doing work proportional to
the size of the input change:

\begin{proposition}
(incremental $\distinct$)
\label{prop-inc_distinct} \\
\noindent
\begin{tabular}{m{3.4cm}m{0cm}m{5cm}}
\begin{tikzpicture}[auto,node distance=1.4cm,>=latex]
    \node[] (input) {$\Delta d$};
    \node[block, right of=input] (d) {$\inc{(\lift{\distinct})}$};
    \node[right of=d] (output) {$\Delta o$};
    \draw[->>] (input) -- (d);
    \draw[->>] (d) -- (output);
\end{tikzpicture} &
$\cong$ &
\begin{tikzpicture}[>=latex]
    \node[] (input) {$\Delta d$};
    \node[block, right of=input] (I) {$\I$};
    \node[block, right of=I] (z) {$\zm$};
    \node[block, below of=z, node distance=.8cm] (H) {$\lift{H}$};
    \node[right of=H] (output) {$\Delta o$};
    \draw[->>] (input) -- node (mid) {} (I);
    \draw[->>] (I) -- (z);
    \draw[->>] (mid.center) |- (H);
    \draw[->>] (z) -- node (i) [right] {} (H);
    \draw[->>] (H) -- (output);
\end{tikzpicture}
\end{tabular}
\noindent where $H: \Z[A] \times \Z[A] \to \Z[A]$ is defined as: \\
$$
H(i, d)[x] \defn
\begin{cases}
-1 & \mbox{if } i[x] > 0 \mbox{ and } (i + d)[x] \leq 0 \\
1  & \mbox{if } i[x] \leq 0 \mbox{ and } (i + d)[x] > 0 \\
0  & \mbox{otherwise} \\
\end{cases}
$$
\end{proposition}

This incremental implementation of $\distinct$ has been known for a
long time; for example, it is called the $\exists$ operator
in~\cite{nikolic-sigmod16}.  This implementation has several
interesting features:

\begin{itemize}
  \item The implementation uses an integral operator $\I$ to
    reconstitute the \emph{entire input set} of the distinct operator
    from the set of changes.  This is the ``top'' input of the $H$
    function.  The implementation needs to maintain the \emph{entire
    input set} (similar to joins) in order to discover whether an item
    is new or not.
  \item Despite this fact, the result of $\distinct$ for an input
    change can still be computed with work proportional to the size of
    the change.  $H$ detects whether the weight of a row in the full
    set is changing sign (from negative to positive on a row
    insertion, and from positive to negative on a deletion) when the
    row appears in a new change.  Only tuples that appear in
    the input change $\Delta d$ can appear in the output change
    $\Delta o$, so the work performed is $O(|\Delta d)|$.
\end{itemize}

The algorithm~\ref{algorithm-inc} reduces the problem of incremental
execution of a query plan to the incremental execution of
sub\-plans/primitive operators.  However, this algorithm can be
applied even if we use a primitive $P$ for which no efficient
incremental version is known: we can always use the inefficient
``brute-force'' implementation given by $\inc{P} = \D \circ \lift{P}
\circ \I$.

\vspace{-3ex}
\textbf{Relational Query Example}\label{sec:relational-example}

%Let's apply the IVM algorithm to a concrete relational SQL query:
\begin{lstlisting}[language=SQL,basicstyle=\small\ttfamily]
CREATE VIEW v AS
SELECT DISTINCT a.x, b.y FROM (
     SELECT t1.x, t1.id FROM t1 WHERE t1.a > 2
) a JOIN (
     SELECT t2.id, t2.y FROM t2 WHERE t2.s > 5
) b ON a.id = b.id
\end{lstlisting}

\vspace{-1ex}

Step 1: Create a \dbsp circuit to represent this query using the
translation rules from Table~\ref{tab:relational}; notice that
this circuit is essentially a dataflow implementation of the query:

\noindent
\begin{tikzpicture}[node distance=1.2cm,>=latex]
    \node[] (t1) {\code{t1}};
    \node[block, right of=t1, node distance=.9cm] (s1) {$\sigma_{a > 2}$};
    \node[block, right of=s1] (d1) {$\distinct$};
    \node[block, right of=d1] (p1) {$\pi_{x, id}$};
    \node[block, right of=p1] (d11) {$\distinct$};
    \node[below of=t1, node distance=1cm] (t2) {\code{t2}};
    \node[block, right of=t2, node distance=.9cm] (s2) {$\sigma_{s > 5}$};
    \node[block, right of=s2] (d2) {$\distinct$};
    \node[block, right of=d2] (p2) {$\pi_{y, id}$};
    \node[block, right of=p2] (d21) {$\distinct$};
    \node[below of=d11, node distance=.5cm] (mid) {};
    \node[block, right of=mid, node distance=.8cm] (j) {$\bowtie_{id = id}$};
    \node[block, right of=j] (p) {$\pi_{x, y}$};
    \node[block, right of=p] (d) {$\distinct$};
    \node[right of=d, node distance=.9cm] (V) {\code{V}};
    \draw[->] (t1) -- (s1);
    \draw[->] (s1) -- (d1);
    \draw[->] (d1) -- (p1);
    \draw[->] (p1) -- (d11);
    \draw[->] (t2) -- (s2);
    \draw[->] (s2) -- (d2);
    \draw[->] (d2) -- (p2);
    \draw[->] (p2) -- (d21);
    \draw[->] (d11) -| (j);
    \draw[->] (d21) -| (j);
    \draw[->] (j) -- (p);
    \draw[->] (p) -- (d);
    \draw[->] (d) -- (V);
\end{tikzpicture}

Step 2: apply rules to eliminate $\distinct$ operators, producing an
equivalent circuit: (from now on we omit the subscripts to save
space):

%\noindent
%\begin{tikzpicture}[node distance=1.2cm,>=latex]
%    \node[] (t1) {\code{t1}};
%    \node[block, right of=t1, node distance=.9cm] (s1) {$\sigma_{a > 2}$};
%    \node[block, right of=s1] (p1) {$\pi_{x, id}$};
%    \node[block, right of=p1] (d11) {$\distinct$};
%    \node[below of=t1, node distance=1cm] (t2) {\code{t2}};
%    \node[block, right of=t2, node distance=.9cm] (s2) {$\sigma_{s > 5}$};
%    \node[block, right of=s2] (p2) {$\pi_{y, id}$};
%    \node[block, right of=p2] (d21) {$\distinct$};
%    \node[below of=d11, node distance=.5cm] (mid) {};
%    \node[block, right of=mid, node distance=.8cm] (j) {$\bowtie_{id = id}$};
%    \node[block, right of=j] (p) {$\pi_{x, y}$};
%    \node[block, right of=p] (d) {$\distinct$};
%    \node[right of=d, node distance=1.2cm] (V) {\code{V}};
%    \draw[->] (t1) -- (s1);
%    \draw[->] (s1) -- (p1);
%    \draw[->] (p1) -- (d11);
%    \draw[->] (t2) -- (s2);
%    \draw[->] (s2) -- (p2);
%    \draw[->] (p2) -- (d21);
%    \draw[->] (d11) -| (j);
%    \draw[->] (d21) -| (j);
%    \draw[->] (j) -- (p);
%    \draw[->] (p) -- (d);
%    \draw[->] (d) -- (V);
%\end{tikzpicture}


%\noindent
%\begin{tikzpicture}[node distance=1.2cm,>=latex]
%    \node[] (t1) {\code{t1}};
%    \node[block, right of=t1, node distance=.9cm] (s1) {$\sigma$};
%    \node[block, right of=s1] (p1) {$\pi$};
%    \node[below of=t1, node distance=1cm] (t2) {\code{t2}};
%    \node[block, right of=t2, node distance=.9cm] (s2) {$\sigma$};
%    \node[block, right of=s2] (p2) {$\pi$};
%    \node[below of=p1, node distance=.5cm] (mid) {};
%    \node[block, right of=mid, node distance=.8cm] (j) {$\bowtie$};
%    \node[block, right of=j] (d0) {$\distinct$};
%    \node[block, right of=d0] (p) {$\pi$};
%    \node[block, right of=p] (d) {$\distinct$};
%    \node[right of=d, node distance=1.2cm] (V) {\code{V}};
%    \draw[->] (t1) -- (s1);
%    \draw[->] (s1) -- (p1);
%    \draw[->] (t2) -- (s2);
%    \draw[->] (s2) -- (p2);
%    \draw[->] (p1) -| (j);
%    \draw[->] (p2) -| (j);
%    \draw[->] (j) -- (d0);
%    \draw[->] (d0) -- (p);
%    \draw[->] (p) -- (d);
%    \draw[->] (d) -- (V);
%\end{tikzpicture}
%
%\noindent And again~\ref{prop-distinct-once}:

\noindent
\begin{tikzpicture}[node distance=1cm,>=latex]
    \node[] (t1) {\code{t1}}; \node[block, right of=t1, node
      distance=.9cm] (s1) {$\sigma$}; \node[block, right of=s1] (p1)
         {$\pi$}; \node[below of=t1, node distance=1cm] (t2)
         {\code{t2}}; \node[block, right of=t2, node distance=.9cm]
         (s2) {$\sigma$}; \node[block, right of=s2] (p2) {$\pi$};
         \node[below of=p1, node distance=.5cm] (mid) {}; \node[block,
           right of=mid, node distance=.8cm] (j) {$\bowtie$};
         \node[block, right of=j] (p) {$\pi$}; \node[block, right
           of=p] (d) {$\distinct$}; \node[right of=d, node
           distance=1cm] (V) {\code{V}}; \draw[->] (t1) -- (s1);
         \draw[->] (s1) -- (p1); \draw[->] (t2) -- (s2); \draw[->]
         (s2) -- (p2); \draw[->] (p1) -| (j); \draw[->] (p2) -| (j);
         \draw[->] (j) -- (p); \draw[->] (p) -- (d); \draw[->] (d) --
         (V);
\end{tikzpicture}

This step is found in some traditional database optimizers.  Notice
that some arrows that used to represent sets in the original circuit
may represent multisets in the optimized circuit.

Step 3: lift the circuit to compute over streams; all arrows are
doubled and all functions are lifted:

\noindent
\begin{tikzpicture}[node distance=1cm,>=latex]
    \node[] (t1) {\code{t1}};
    \node[block, right of=t1, node distance=.9cm] (s1) {$\lift{\sigma}$};
    \node[block, right of=s1] (p1) {$\lift{\pi}$};
    \node[below of=t1, node distance=1.2cm] (t2) {\code{t2}};
    \node[block, right of=t2, node distance=.9cm] (s2) {$\lift{\sigma}$};
    \node[block, right of=s2] (p2) {$\lift{\pi}$};
    \node[below of=p1, node distance=.6cm] (mid) {};
    \node[block, right of=mid, node distance=.8cm] (j) {$\lift{\bowtie}$};
    \node[block, right of=j] (p) {$\lift{\pi}$};
    \node[block, right of=p, node distance=1.4cm] (d) {$\lift{\distinct}$};
    \node[right of=d, node distance=1.3cm] (V) {\code{V}};
    \draw[->>] (t1) -- (s1);
    \draw[->>] (s1) -- (p1);
    \draw[->>] (t2) -- (s2);
    \draw[->>] (s2) -- (p2);
    \draw[->>] (p1) -| (j);
    \draw[->>] (p2) -| (j);
    \draw[->>] (j) -- (p);
    \draw[->>] (p) -- (d);
    \draw[->>] (d) -- (V);
\end{tikzpicture}

Step 4: incrementalize circuit, obtaining a circuit that computes over changes;
this circuit receives changes to relations \code{t1} and \code{t2} and for each
such change it produces the corresponding change in the output view \code{V}:

\noindent
\begin{tikzpicture}[node distance=1cm,>=latex]
    \node[] (t1) {$\Delta$\code{t1}};
    \node[block, right of=t1, node distance=.8cm] (I1) {$\I$};
    \node[block, right of=I1, node distance=.9cm] (s1) {$\lift{\sigma}$};
    \node[block, right of=s1] (p1) {$\lift{\pi}$};
    \node[below of=t1, node distance=1.2cm] (t2) {$\Delta$\code{t2}};
    \node[block, right of=t2, node distance=.8cm] (I2) {$\I$};
    \node[block, right of=I2, node distance=.9cm] (s2) {$\lift{\sigma}$};
    \node[block, right of=s2] (p2) {$\lift{\pi}$};
    \node[below of=p1, node distance=.6cm] (mid) {};
    \node[block, right of=mid, node distance=.7cm] (j) {$\lift{\bowtie}$};
    \node[block, right of=j] (p) {$\lift{\pi}$};
    \node[block, right of=p, node distance=1.4cm] (d) {$\lift{\distinct}$};
    \node[block, right of=d, node distance=1.3cm] (D) {$\D$};
    \node[right of=D, node distance=.8cm] (V) {$\Delta$\code{V}};
    \draw[->>] (t1) -- (I1);
    \draw[->>] (I1) -- (s1);
    \draw[->>] (s1) -- (p1);
    \draw[->>] (t2) -- (I2);
    \draw[->>] (I2) -- (s2);
    \draw[->>] (s2) -- (p2);
    \draw[->>] (p1) -| (j);
    \draw[->>] (p2) -| (j);
    \draw[->>] (j) -- (p);
    \draw[->>] (p) -- (d);
    \draw[->>] (d) -- (D);
    \draw[->>] (D) -- (V);
\end{tikzpicture}

Step 5: apply the chain rule to rewrite the circuit as a composition
of incremental operators; notice the use of $\inc{.}$ for all
operators:

\noindent
\begin{tikzpicture}[node distance=1.6cm,>=latex]
    \node[] (t1) {$\Delta$\code{t1}};
    \node[block, right of=t1, node distance=1.2cm] (s1) {$\inc{(\lift{\sigma})}$};
    \node[block, right of=s1, node distance=1.4cm] (p1) {$\inc{(\lift{\pi})}$};
    \node[below of=t1, node distance=1.2cm] (t2) {$\Delta$\code{t2}};
    \node[block, right of=t2, node distance=1.2cm] (s2) {$\inc{(\lift{\sigma})}$};
    \node[block, right of=s2, node distance=1.4cm] (p2) {$\inc{(\lift{\pi})}$};
    \node[below of=p1, node distance=.6cm] (mid) {};
    \node[block, right of=mid, node distance=.6cm] (j) {$\inc{(\lift{\bowtie})}$};
    \node[block, right of=j,node distance=1.4cm] (p) {$\inc{(\lift{\pi})}$};
    \node[block, right of=p,node distance=1.7cm] (d) {$\inc{(\lift{\distinct})}$};
    \node[right of=d, node distance=1.5cm] (V) {$\Delta$\code{V}};
    \draw[->>] (t1) -- (s1);
    \draw[->>] (s1) -- (p1);
    \draw[->>] (t2) -- (s2);
    \draw[->>] (s2) -- (p2);
    \draw[->>] (p1) -| (j);
    \draw[->>] (p2) -| (j);
    \draw[->>] (j) -- (p);
    \draw[->>] (p) -- (d);
    \draw[->>] (d) -- (V);
\end{tikzpicture}

Use the linearity of $\sigma$ and $\pi$ to simplify this circuit (notice that
all linear operators no longer have a $\inc{\cdot}$):

\noindent
\begin{tikzpicture}[node distance=1cm,>=latex]
    \node[] (t1) {$\Delta$\code{t1}};
    \node[block, right of=t1, node distance=1cm] (s1) {$\lift{\sigma}$};
    \node[block, right of=s1] (p1) {$\lift{\pi}$};
    \node[below of=t1, node distance=1.2cm] (t2) {$\Delta$\code{t2}};
    \node[block, right of=t2, node distance=1cm] (s2) {$\lift{\sigma}$};
    \node[block, right of=s2] (p2) {$\lift{\pi}$};
    \node[below of=p1, node distance=.6cm] (mid) {};
    \node[block, right of=mid, node distance=.8cm] (j) {$\inc{(\lift{\bowtie})}$};
    \node[block, right of=j, node distance=1.2cm] (p) {$\lift{\pi}$};
    \node[block, right of=p, node distance=1.6cm] (d) {$\inc{(\lift{\distinct})}$};
    \node[right of=d, node distance=1.6cm] (V) {$\Delta$\code{V}};
    \draw[->>] (t1) -- (s1);
    \draw[->>] (s1) -- (p1);
    \draw[->>] (t2) -- (s2);
    \draw[->>] (s2) -- (p2);
    \draw[->>] (p1) -| (j);
    \draw[->>] (p2) -| (j);
    \draw[->>] (j) -- (p);
    \draw[->>] (p) -- (d);
    \draw[->>] (d) -- (V);
\end{tikzpicture}

Finally, replace the incremental join and the incremental $\distinct$,
with their incremental implementations, obtaining the following
circuit (we have used a slightly different expansion for the join than
the one shown previously; this one only contains two integrators):

\noindent
\begin{tikzpicture}[node distance=.8cm,>=latex]
    \node[] (t1) {$\Delta$\code{t1}};
    \node[block, right of=t1] (s1) {$\lift{\sigma}$};
    \node[block, right of=s1] (p1) {$\lift{\pi}$};
    \node[below of=t1, node distance=.8cm] (t2) {$\Delta$\code{t2}};
    \node[block, right of=t2] (s2) {$\lift{\sigma}$};
    \node[block, right of=s2] (p2) {$\lift{\pi}$};

    % join expansion
      \node[block, right of=p1] (jI1) {$\I$};
      \node[block, right of=p2] (jI2) {$\I$};
      \draw[->>] (p1) -- (jI1);
      \draw[->>] (p2) -- (jI2);
      \node[block, right of=jI2] (ZI2) {$\zm$};
      \draw[->>] (jI2) -- (ZI2);
      \node[block, right of=jI1] (DI1) {$\lift\bowtie$};
      \node[block, right of=ZI2, node distance=1cm] (DI2) {$\lift\bowtie$};
      \draw[->>] (jI1) -- (DI1);
      \draw[->>] (ZI2) -- (DI2);
      \node[block, circle, above of=DI2, inner sep=0cm] (sum) {$+$};
      \draw[->>] (DI1) -- (sum);
      \draw[->>] (DI2) -- (sum);
      \draw[->>] (p1) -- (DI2);
      \draw[->>] (p2) -- (DI1);

    \node[block, right of=sum] (p) {$\lift{\pi}$};
    \draw[->>] (sum) -- (p);
    \node[block, right of=p] (Id) {$\I$};
    \node[block, right of=Id, node distance=1cm] (zd) {$\zm$};
    \node[block, below of=zd] (H) {$\lift{H}$};
    \node[right of=H, node distance=1cm] (V) {$\Delta$\code{V}};
    \draw[->>] (t1) -- (s1);
    \draw[->>] (s1) -- (p1);
    \draw[->>] (t2) -- (s2);
    \draw[->>] (s2) -- (p2);
    \draw[->>] (p) -- node (tapp) {} (Id);
    \draw[->>] (Id) -- (zd);
    \draw[->>] (zd) -- (H);
    \draw[->>] (tapp.center) |- (H);
    \draw[->>] (H) -- (V);
\end{tikzpicture}

Notice that the resulting circuit contains three integration
operations: two from the join, and one from the $\distinct$.  It also
contains two join operators.  However, the work performed by each
operator for each new input is proportional to the size of its input
change.


\vspace{-3ex}
\subsection{Complexity of incremental circuits}\label{sec:complexity}

Incremental circuits are efficient.  We evaluate the cost of a circuit
while processing the $t$-th input change.  Even if $Q$ is a pure
function, $\inc{Q}$ is actually a streaming system, with internal
state.  This state is stored entirely in the delay operators $\zm$,
some of which appear in $\I$ and $\D$ operators\footnote{For
standard relational queries, after applying step 5 of the algorithm
there will be no $\D$ operators left in the circuit.}.  The result
produced by $\inc{Q}$ on the $t$-th input depends in general not only
on the new $t$-th input, but also on all prior inputs it has received.

We argue that each operator in the incremental version of a circuit is
efficient in terms of work and space.  We make the standard IVM
assumption that the input changes \emph{of each operator} are small:
$|\Delta DB[t]| \ll |DB[t]| = |(\I(\Delta DB))[t]|$\footnote{One can
write SQL programs where each output row can change even for very
small input changes, so this assumption does not always hold in
practice, but even in this case, the asymptotic work performed by the
incremental query is not worse than the work of the original query.}.
We omit the time-index $[t]$ for readability in the rest of the
section when it is unambiguous; all formulas below hold for every time
step.

An unoptimized incremental operator $\inc{Q} = \D \circ Q \circ \I$
evaluates query $Q$ on the whole database $DB$, the integral of the
input stream: $DB = \I(\Delta DB)$; hence its time complexity is the
same as that of the non-incremental evaluation of $Q$.  In addition,
each of the $\I$ and $\D$ operators uses $O(|DB|)$ memory.

Step (5) of the incrementalization algorithm applies the optimizations
described in \secref{sec:incremental}; these reduce the time
complexity of each unary operator to $O(|\Delta DB|)$.  Bilinear
operators, including join, can be evaluated in time $O(|DB| \times
|\Delta DB|)$; both of these are a factor of $|DB| / |\Delta DB|$
better than full re-evaluation.  For example, Theorem~\ref{linear},
allows evaluating $\inc{S}$, where $S$ is a linear operator, in time
$O(|\Delta DB|)$.  The $\I$ operator can also be evaluated in
$O(|\Delta DB|)$ time, because all values that appear in the output of
$\I(\Delta DB)[t]$ must be present in current input change $\Delta
DB[t]$\footnote{Assuming concatenation is constant-time; in our
implementation the cost of $\I$ is $O(|\Delta DB| \log(|DB|))$ per
time step.}.  Similarly, while the $\distinct$ operator is not linear,
$\inc{(\lift{\distinct})}$ can also be evaluated in $O(|\Delta DB|)$
according to Proposition~\ref{prop-inc_distinct}.

The space complexity of linear operators is zero (O(1)), since they
store no data persistently.  The space complexity of operators such as
$\inc{(\lift{\distinct})}$, $\inc{(\lift{\bowtie})}$, $\I$, and $\D$
is $O(|DB|)$.  They need to store their input or output relations in
full (moreover, some of these are intermediate results, which may be
even larger than the input database).
